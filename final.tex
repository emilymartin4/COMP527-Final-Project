\documentclass{article}


%% Extend math functions
\usepackage{amsmath, amsfonts, mathtools, amsthm, amssymb}

%% Macros to support stacking objects vertically
\usepackage{stackengine}

%% Don't indent paragraphs, instead add more spacing between them
\usepackage{parskip}

\usepackage{mathdots}

%% Graphical
\usepackage{tikz}
\usepackage{tikz-cd}
\usetikzlibrary{automata, positioning, arrows, shapes}
\tikzset{ %https://www3.nd.edu/~kogge/courses/cse30151-fa17/Public/other/tikz_tutorial.pdf
    ->, % makes the edges directed
    >=stealth, % makes the arrow heads bold
    node distance=2cm, % specifies the minimum distance between two nodes. Change if necessary.
    every state/.style={thick, fill=gray!10}, % sets the properties for each ’state’ node
    initial text=\( \), % sets the text that appears on the start arrow
}

\usepackage{graphicx}
\usepackage{import}
\usepackage{xifthen}
\usepackage{pdfpages}
\usepackage{transparent}

%% Cancel var commands
\usepackage{cancel}

%% Wrap text around figures and other graphics
\usepackage{wrapfig}

%% lorem ipsum generator
\usepackage{lipsum}

%% Caption controls
\usepackage{caption}

%% Prog-Code typeset
\usepackage{listings}
\lstset{
    showstringspaces=false
}

%% Linked TOC
\usepackage{hyperref}

%% Get name instead of number of reference via \nameref{...}
\usepackage{nameref}

%% Table styling
\usepackage{booktabs}

%% For \FloatBarrier command
\usepackage{placeins}

%% Additional math symbols + theoretical computer science symbols
\usepackage{stmaryrd}

%% Proof trees
\usepackage{bussproofs}
\usepackage{bussproofs-extra}

%>> -------------------- Redeclare math op --------------------  <<%%
%% https://tex.stackexchange.com/questions/175251/how-to-redefine-a-command-using-declaremathoperator
\makeatletter
\newcommand\RedeclareMathOperator{%
    \@ifstar{\def\rmo@s{m}\rmo@redeclare}{\def\rmo@s{o}\rmo@redeclare}%
}
% this is taken from \renew@command
\newcommand\rmo@redeclare[2]{%
    \begingroup \escapechar\m@ne\xdef\@gtempa{{\string#1}}\endgroup
    \expandafter\@ifundefined\@gtempa
        {\@latex@error{\noexpand#1undefined}\@ehc}%
        \relax
    \expandafter\rmo@declmathop\rmo@s{#1}{#2}}
% This is just \@declmathop without \@ifdefinable
\newcommand\rmo@declmathop[3]{%
    \DeclareRobustCommand{#2}{\qopname\newmcodes@#1{#3}}%
}
\@onlypreamble\RedeclareMathOperator
\makeatother
\usepackage{titlesec}
\titleformat*{\paragraph}{\normalfont\itshape}

%%>> -------------------- Checklist --------------------  <<%%
\usepackage{pifont}
\usepackage{enumitem}
\newlist{checklist}{itemize}{2}
\setlist[checklist]{label=\(\square\)}
\newcommand{\done}{\rlap{\(\square\)}{\raisebox{1.5pt}{\large\hspace{1pt}\ding{51}}}\hspace{-2.5pt}}

%%>> -------------------- QOL Commands --------------------  <<%%

\DeclareMathOperator\N{\mathbb{N}}
\DeclareMathOperator\R{\mathbb{R}}
\DeclareMathOperator\Z{\mathbb{Z}}
\let\p\relax
\RedeclareMathOperator\P{\mathbb{P}}
\DeclareMathOperator\E{\mathbb{E}}
\renewcommand\O{\emptyset}
\DeclareMathOperator\Q{\mathbb{Q}}
\DeclareMathOperator\C{\mathbb{C}}
\DeclareMathOperator\infin{\infty}

\DeclareMathOperator\sub{\subset}
\DeclareMathOperator\sube{\subseteq}
\DeclareMathOperator\nsube{\nsubseteq}
\DeclareMathOperator\unif{\text{Unif}}

\newcommand{\howe}[1][\tau]{\preccurlyeq_{#1}^{\mathcal{H}}}
\newcommand{\osim}[1][\tau]{\preccurlyeq_{#1}^{\circ}}
\newcommand{\asim}[1][\tau]{\preccurlyeq_{#1}}


%% https://tex.stackexchange.com/questions/85059/proof-environment-line-break-after-the-proof
\newcommand{\forceNewline}{\( \)\par\nobreak\ignorespaces}%

%% https://tex.stackexchange.com/questions/50804/explicit-space-character
\newcommand\visiblespace[1][.3em]{%
    \mbox{\kern.06em\vrule height.3ex}%
    \vbox{\hrule width#1}%
    \hbox{\vrule height.3ex}%
}
\usepackage{geometry} % Required for adjusting page dimensions and margins
\geometry{margin=2cm}

%%>> -------------------- Theorem Environments --------------------  <<%%


% \usepackage{thmtools}
% \usepackage{thm-restate}
% \usepackage[framemethod=TikZ]{mdframed}
% \mdfsetup{skipabove=1em,skipbelow=0em, innertopmargin=12pt, innerbottommargin=8pt}

% \theoremstyle{definition}

% \declaretheoremstyle[headfont=\bfseries\sffamily, bodyfont=\normalfont, mdframed={ nobreak } ]{thmgreenbox}
% \declaretheoremstyle[headfont=\bfseries\sffamily, bodyfont=\normalfont, mdframed={ nobreak } ]{thmredbox}
% \declaretheoremstyle[headfont=\bfseries\sffamily, bodyfont=\normalfont]{thmbluebox}
% \declaretheoremstyle[headfont=\bfseries\sffamily, bodyfont=\normalfont]{thmblueline}
% \declaretheoremstyle[headfont=\bfseries\sffamily, bodyfont=\normalfont, numbered=no, mdframed={ rightline=false, topline=false, bottomline=false, }, qed=\qedsymbol ]{thmproofbox}
% \declaretheoremstyle[headfont=\bfseries\sffamily, bodyfont=\normalfont, numbered=no, mdframed={ nobreak, rightline=false, topline=false, bottomline=false } ]{thmexplanationbox}


% \declaretheorem[numbered=no, style=thmgreenbox, name=Definition]{definition}
% \declaretheorem[numberwithin=chapter, style=thmredbox, name=Theorem]{theorem}
% \declaretheorem[sibling=theorem, style=thmredbox, name=Corollary]{corollary}
% \declaretheorem[sibling=theorem, style=thmredbox, name=Proposition]{prop}
% \declaretheorem[sibling=theorem, style=thmredbox, name=Lemma]{lemma}
% \declaretheorem[numberwithin=chapter, style=thmbluebox, name=Example]{example}
% \declaretheorem[sibling=example, style=thmbluebox, name=Nonexample]{nonexample}
% \declaretheorem[numbered=no, style=thmblueline, name=Remark]{remark}
% \declaretheorem[numbered=no, style=thmredbox, name=Observation]{observation}
% \declaretheorem[numbered=no, style=thmredbox, name=Postulate]{postulate}
% \declaretheorem[sibling=theorem, style=thmredbox, name=Observation]{observationnum}
% \declaretheorem[numbered=no, style=thmblueline, name=Axiom]{axiom}
% \declaretheorem[numbered=no, style=thmblueline, name=Conjecture]{conjecture}
% \declaretheorem[numbered=no, style=thmblueline, name=Notation]{notation}
% \declaretheorem[style=thmbluebox,  numbered=no, name=Exercise]{exercise}

%>> -------------------- Figures --------------------  <<%%

%% For non-tuftebook styled sheets this is almost identical to \largeinkfig
\newcommand{\inkfig}[2][1]{%
    \def\svgwidth{#1\textwidth}%
    \import{./figures}{#2.pdf_tex}%
}

\newcommand{\largeinkfig}[2][0.90]{%
    \def\svgwidth{#1\paperwidth}%
    \import{./figures}{#2.pdf_tex}%
}


\newcommand{\smallinkfig}[2][0.7]{%
    \def\svgwidth{\columnwidth}
    \resizebox{#1\textwidth}{!}{\import{./figures}{#2.pdf_tex}}
}


\title{W2025 - COMP527 Final}
\author{L, Emily, Alex}
\date{\today}

\graphicspath{{./nifigures/}}

\begin{document}
\maketitle

\paragraph{Rules of Inference} \begin{center}
    \AxiomC{\(\Gamma\vdash M_1:\sigma\)}
    \AxiomC{\(\Gamma\vdash M_2:\sigma'\)}
    \RightLabel{\(\vdash~\text{pair}\)}\BinaryInfC{\(\Gamma\vdash \left\langle M_1,M_2 \right\rangle :\sigma\times\sigma'\)}
    \DisplayProof
\end{center}
\begin{center}
    \AxiomC{\(\Gamma\vdash P:\sigma\times\sigma'\)}
    \RightLabel{\(\vdash~\text{fst}\)}\UnaryInfC{\(\Gamma\vdash~\text{fst}(P):\sigma\)}
    \DisplayProof
    \(\quad\)
    \AxiomC{\(\Gamma\vdash P:\sigma\times\sigma'\)}
    \RightLabel{\(\vdash~\text{snd}\)}\UnaryInfC{\(\Gamma\vdash~\text{snd}(P):\sigma'\)}
    \DisplayProof
\end{center}
\begin{center}
    \AxiomC{\(\Gamma\vdash P\Downarrow \left\langle M_1,M_2 \right\rangle \)}
    \AxiomC{\(\Gamma\vdash M_1\Downarrow c\)}
    \RightLabel{\(\Downarrow~\text{fst}\)}\BinaryInfC{\(\Gamma\vdash \text{fst}(P)\Downarrow c\)}
    \DisplayProof
    \(\quad\)
    \AxiomC{\(\Gamma\vdash P\Downarrow \left\langle M_1,M_2 \right\rangle \)}
    \AxiomC{\(\Gamma\vdash M_2\Downarrow c\)}
    \RightLabel{\(\Downarrow~\text{snd}\)}\BinaryInfC{\(\Gamma\vdash \text{snd}(P)\Downarrow c\)}
    \DisplayProof
\end{center}

\paragraph{Proposed Howe's Inference Rules} \begin{center}
    \AxiomC{\(\Gamma\vdash m\howe[\tau\times\tau'] m'\)}
    \AxiomC{\(\Gamma\vdash~\text{fst}~m'\osim n\)}
    \RightLabel{\(\text{hfst}\)}\BinaryInfC{\(\Gamma\vdash~\text{fst}(m)\howe n\)}
    \DisplayProof
    \(\quad\)
    \AxiomC{\(\Gamma\vdash m\howe[\tau\times\tau'] m'\)}
    \AxiomC{\(\Gamma\vdash~\text{snd}~m'\osim[\tau'] n\)}
    \RightLabel{\(\text{hsnd}\)}\BinaryInfC{\(\Gamma\vdash~\text{snd}(m)\howe[\tau'] n\)}
    \DisplayProof
\end{center}

\begin{center}
    \AxiomC{\(\Gamma\vdash m_1\howe m_1'\)}
    \AxiomC{\(\Gamma\vdash m_2\howe[\tau'] m_2'\)}
    \AxiomC{\(\Gamma\vdash \left\langle m_1',m_2' \right\rangle \osim[\tau\times\tau']n\)}
    \RightLabel{\(\text{hpair}\)}\TrinaryInfC{\(\Gamma\vdash \left\langle m_1,m_2 \right\rangle \howe[\tau\times\tau'] n\)}
    \DisplayProof
\end{center}



\paragraph{Objective} Prove the following theorems (from Howe's paper) on our extension. \begin{checklist}
    \item[\done] Extend Definition 1 Applicative Simulation to pairs
    \item[\done] Prove Theorem 1 (Reflexivity of Applicative Similarity, \(\forall m:\tau,m\asim m\)) holds for our extension
    \item[\done] Extend Definition 2 Compatible relation to pairs.
    \item[\done] Prove Lemma 6 (Substitutivity of the Howe Relation) holds for our extension
    \item Prove Theorem 3 (Howe's Relation Coincide with Similarity, \(\Gamma\vdash p\howe q\iff \Gamma\vdash p\osim q\)). Achieved by first proving the lemmata presented below, then concluding the theorem. \begin{checklist}
        \item[\done] Semi-transitivity; \((m\osim m')\wedge(m'howe m'')\implies m\howe m''\)
        \item[\done] Reflexive
        \item Compatibility; extension of definition 2 holds
        \item Open similarity is contained in Howe
        \item[\done] Howe relation is substitutive
        \item Howe relation mimics the Simulation condition, extend for pairs. If \(\left\langle m_1,m_2 \right\rangle \howe[\tau\times\tau'] n\) then \(n\Downarrow \left\langle p_1,p_2 \right\rangle \) with \(m_1\howe p_1\) and \(m_2\howe[\tau'] p_2\)
        \item Downward closure, \((p\howe q)\wedge (p\Downarrow v)\implies v\howe q\)
        \item \(p\howe q \implies p\asim q\)
    \end{checklist}
\end{checklist}
For proofs where \(\text{fst},\text{snd}\) are similar, we use \(\text{fst}\) as the representative case.

\paragraph{Definition 1. Applicative Simulation} We extend to pairs by adding the following, \begin{itemize}
    \item If \(m ~\mathcal{R}_{\tau\times\tau'}~n\) then \(m\Downarrow \left\langle m_f,m_s \right\rangle \) entails there are terms \(m_f',m_s'\) s.t. \(n\Downarrow \left\langle m_f',m_s' \right\rangle \) for which \(m_f ~\mathcal{R}_{\tau}~m_f'\) and \(m_s ~\mathcal{R}_{\tau'}~m_s'\)
\end{itemize}

\paragraph{Theorem 1. Reflexitivity of Applicative Similarity} We show \(\forall m:\tau,m\asim m\) holds with our extension of applicative simulation by adding a case for pairs. We note \(S_{\tau}\) to be the family \(\left\{ (m,m) : \dot\vdash m:\tau \right\} \). Assume \(m~\mathcal{S}_{\tau\times\tau'}~m\) and \(m\Downarrow \left\langle m_f,m_s \right\rangle \). Pick \(m_f',m_s'\) to be \(m_f,m_s\). By definition of simulation, \(m_f~\mathcal{S}_{\tau}~m_f\) and \(m_s~\mathcal{S}_{\tau'}~m_s\) \hfill \(\square\)

\paragraph{Definition 2. Compatible relation} We extend the definition of a compatible relation with the following, \begin{itemize}
    \item[(C7)] \(\Gamma\vdash m_1~\mathcal{R}_{\tau}~m_2\) and \(\Gamma\vdash n_1~\mathcal{R}_{\tau'}~n_2\) entails \(\Gamma\vdash \left\langle m_1,n_1 \right\rangle ~\mathcal{R}_{\tau\times\tau'}~\left\langle m_2,n_2 \right\rangle \)
    \item[(C8)] \(\Gamma\vdash m_1~\mathcal{R}_{\tau\times\tau'}~m_2\) entails \(\Gamma\vdash \text{fst}(m_1)~\mathcal{R}_{\tau}~\text{fst}(m_2)\)
    \item[(C9)] \(\Gamma\vdash m_1~\mathcal{R}_{\tau\times\tau'}~m_2\) entails \(\Gamma\vdash \text{snd}(m_1)~\mathcal{R}_{\tau'}~\text{snd}(m_2)\)
\end{itemize}


\paragraph{Lemma 6. Substitutivity of the Howe Relation.} Suppose we have \(\Gamma\vdash m_1 \howe m_2 \text{ and } \Psi \vdash \sigma_1 \howe[\Gamma] \sigma_2; \text{ then } \Psi \vdash [\sigma_1]m_1 \howe{}
[\sigma_2]m_2\). Proof by induction on the derivations of \( \Gamma \vdash m_1 \howe m_2\). We extend this to the derivations of the Howe relation for pairs.

\textbf{Case}
\begin{prooftree}
    \AxiomC{\(\mathcal{D}_1\)}
    \noLine
    \UnaryInfC{\(\Gamma \vdash m_1 \howe m_1'\)}
    \AxiomC{\(\mathcal{D}_2\)}
    \noLine
    \UnaryInfC{\(\Gamma \vdash m_2 \howe[\tau'] m_2'\)}
    \AxiomC{\(\mathcal{D}_3\)}
    \noLine
    \UnaryInfC{\(\Gamma \vdash \left\langle m_1', m_2' \right\rangle  \osim[\tau \times \tau'] n\)}
    \RightLabel{hpair}
    \LeftLabel{\(\mathcal{D} = \)}
    \TrinaryInfC{\(\Gamma \vdash  \left\langle m_1, m_2 \right\rangle  \howe[\tau \times \tau'] n\)}
\end{prooftree}

WTS: \(\Psi \vdash [\sigma_1] \left\langle m_1, m_2 \right\rangle  \howe[\tau \times \tau'] [\sigma_2]n\) \begin{align*}
& \Psi \vdash [\sigma_1]m_1 \howe{} [\sigma_2]m_1' &\text{ by IH on } \mathcal{D}_1 \\
& \Psi \vdash [\sigma_1]m_2 \howe{} [\sigma_2]m_2' &\text{ by IH on } \mathcal{D}_2 \\
& \Psi \vdash [\sigma_2]\left\langle m_1', m_2' \right\rangle  \osim[\tau \times \tau'] [\sigma_2]n & \text{ by cus on } \mathcal{D}_3 &\\
& [\sigma_2] \left\langle m_1', m_2' \right\rangle  \; = \; \left\langle [\sigma_2]m_1', [\sigma_2]m_2' \right\rangle  & \text{ by def. of substitution }&
\end{align*}

Construct: \begin{prooftree}
    \AxiomC{}
    \noLine
    \UnaryInfC{\(\Psi \vdash [\sigma_1]m_1 \howe{} [\sigma_2]m_1'\)}
    \AxiomC{}
    \noLine
    \UnaryInfC{\(\Psi \vdash [\sigma_1]m_2 \howe[\tau'] [\sigma_2]m_2'\)}
    \AxiomC{}
    \noLine
    \UnaryInfC{\(\Psi \vdash \left\langle [\sigma_2] m_1', [\sigma_2]m_2' \right\rangle  \osim[\tau \times \tau'] [\sigma_2]n\)}
    \RightLabel{hpair}
    \TrinaryInfC{\(\Psi \vdash \left\langle [\sigma_1]m_1, [\sigma_1]m_2 \right\rangle  \howe[\tau \times \tau'] [\sigma_2]n\)}
\end{prooftree}

By the conclusion of the constructed derivation, \(\Psi \vdash [\sigma_1]\left\langle m_1, m_2 \right\rangle  \howe[\tau \times \tau'] [\sigma_2]n\).

\textbf{Case} \begin{center}
    \(\mathcal{D}=\) \AxiomC{\(\mathcal{D}_1\)}
    \noLine\UnaryInfC{\(\Gamma\vdash m\howe[\tau\times\tau']m'\)}
    \AxiomC{\(\mathcal{D}_2\)}
    \noLine\UnaryInfC{\(\Gamma\vdash \text{fst}(m')\osim n\)}
    \RightLabel{\(\text{hfst}\)}\BinaryInfC{\(\Gamma\vdash \text{fst}(m)\howe n\)}
    \DisplayProof
\end{center} \begin{align*}
    \Psi&\vdash[\sigma_1]m\howe[\tau\times\tau'] [\sigma_2]m'&\text{By IH on }\mathcal{D}_1~(\text{denote as }\mathcal{F}_1)\\
    \Psi&\vdash[\sigma_2]\text{fst}(m')\osim{} [\sigma_2]n&\text{By (cus) on }\mathcal{D}_2\\
    \Psi&\vdash\text{fst}([\sigma_2]m')\osim{} [\sigma_2]n&\text{By Substitution }(\text{denote as }\mathcal{F}_2)\\
    \Psi&\vdash\text{fst}([\sigma_1]m)\howe{} [\sigma_2]n&\text{By hfst on } \mathcal{F}_1,\mathcal{F}_2\\
    \Psi&\vdash[\sigma_1]\text{fst}(m)\howe{} [\sigma_2]n&\text{By Substitution }
\end{align*} \hfill \(\square\)


\section*{Theorem 3}

\paragraph{Lemmata 1} Semi-transitivity of Howe's relation. \((m\howe n')\wedge(n'\osim m'')\implies m\howe m''\)

We define \(m=\left\langle m_1,m_2 \right\rangle, n=\left\langle n_1,n_2 \right\rangle, n'=\left\langle n_1',n_2' \right\rangle \).

\textbf{Case.} With assumption \(n'\osimp m''\) \begin{center}
    \(\mathcal{D}=\) \AxiomC{\(\mathcal{D}_1\)}
    \noLine\UnaryInfC{\(\Gamma\vdash m_1\howe n_1\)}
    \AxiomC{\(\mathcal{D}_2\)}
    \noLine\UnaryInfC{\(\Gamma\vdash m_2\howe[\tau'] n_2\)}
    \AxiomC{\(\mathcal{D}_3\)}
    \noLine\UnaryInfC{\(\Gamma\vdash \left\langle n_1, n_2 \right\rangle \osim[\tau\times\tau'] n' \)}
    \RightLabel{\(\text{hpair}\)}\TrinaryInfC{\(\Gamma\vdash \left\langle m_1,m_2 \right\rangle \howep n'\)}
    \DisplayProof
\end{center} \begin{align*}
    n&\osimp m''&&\text{By Transitivity of }\osimp~\text{on }\mathcal{D}_3~\text{and Assumption}\\
    \Gamma\vdash \left\langle m_1,m_2 \right\rangle &\howep m''&&\text{By hpair on } \mathcal{D}_1,\mathcal{D}_2,~\text{and above}
\end{align*}

\textbf{Case.} With assumption \(\text{fst}(n')\osim \text{fst}(m'')\) \begin{center}
    \(\mathcal{D}=\) \AxiomC{\(\mathcal{D}_1\)}
    \noLine\UnaryInfC{\(\Gamma\vdash m\howep n\)}
    \AxiomC{\(\mathcal{D}_2\)}
    \noLine\UnaryInfC{\(\Gamma\vdash \text{fst}(n)\osim \text{fst}(n')\)}
    \RightLabel{\(\text{hfst}\)}\BinaryInfC{\(\Gamma\vdash \text{fst}(m)\howe \text{fst}(n')\)}
    \DisplayProof
\end{center} \begin{align*}
    % \text{fst}(n')&\osim \text{fst}(m'')&&\text{By (C8) on Assumption}\\
    \text{fst}(n)&\osim \text{fst}(m'')&&\text{By Transitivity of }\osim~\text{on }\mathcal{D}_2~\text{and assumption}\\
    \Gamma\vdash \text{fst}(m)&\howe \text{fst}(m'')&&\text{By hfst on } \mathcal{D}_1~\text{and above}
\end{align*} \hfill \(\square\)

\paragraph{Lemmata 2} Reflexivity of Howe's relation. If \(\Gamma\vdash m:\tau\) then \(\Gamma\vdash m\howe m\). Proof by induction on typing,

\textbf{Case.} \begin{center}
    \(\mathcal{D}=\) \AxiomC{\(\mathcal{D}_1\)}
    \noLine\UnaryInfC{\(\Gamma\vdash M_1:\sigma\)}
    \AxiomC{\(\mathcal{D}_2\)}
    \noLine\UnaryInfC{\(\Gamma\vdash M_2:\sigma'\)}
    \RightLabel{\(\vdash\text{pair}\)}\BinaryInfC{\(\Gamma\vdash \left\langle M_1,M_2 \right\rangle :\sigma\times\sigma'\)}
    \DisplayProof
\end{center} \begin{align*}
    \Gamma&\vdash M_1\howe[\sigma] M_1 &\text{By IH on } \mathcal{D}_1~(\text{denote as }\mathcal{F}_1)\\
    \Gamma&\vdash M_2\howe[\sigma] M_2 &\text{By IH on } \mathcal{D}_2~(\text{denote as }\mathcal{F}_2)\\
    \Gamma&\vdash \left\langle M_1,M_2 \right\rangle \osim[\sigma\times\sigma'] \left\langle M_1,M_2 \right\rangle&\text{By reflexivity of }\osim[\sigma\times\sigma']~(\text{denote as } \mathcal{F}_3) \\
    \Gamma&\vdash \left\langle M_1,M_2 \right\rangle \howe[\sigma\times\sigma'] \left\langle M_1,M_2 \right\rangle&\text{By hpair on } \mathcal{F}_1,\mathcal{F}_2,\mathcal{F}_3
\end{align*}

\textbf{Case.} \begin{center}
    \(\mathcal{D}=\) \AxiomC{\(\mathcal{D}_1\)}
    \noLine\UnaryInfC{\(\Gamma\vdash P:\sigma\times\sigma'\)}
    \RightLabel{\(\vdash\text{fst}\)}\UnaryInfC{\(\Gamma\vdash \text{fst}(P):\sigma\)}
    \DisplayProof
\end{center} \begin{align*}
    \Gamma&\vdash P\howe[\sigma\times\sigma'] P&\text{By IH on } \mathcal{D}_1~(\text{denote as } \mathcal{F}_1)\\
    \Gamma&\vdash \text{fst}(P)\osim[\sigma]\text{fst}(P)&\text{By reflexivity of }\osim[\sigma]~(\text{denote as } \mathcal{F}_2)\\
    \Gamma&\vdash \text{fst}(P)\howe[\sigma]\text{fst}(P)&\text{By hfst on } \mathcal{F}_1,\mathcal{F}_2
\end{align*} \hfill \(\square\)

\paragraph{Lemmata 3} Compatibility, (C7-C9) holds. \begin{itemize}
    \item[C7]
    \item[C8]
    \item[C9]
\end{itemize} \hfill \(\square\)

\end{document}