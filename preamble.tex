
%% Extend math functions
\usepackage{amsmath, amsfonts, mathtools, amsthm, amssymb}

%% Macros to support stacking objects vertically
\usepackage{stackengine}

%% Don't indent paragraphs, instead add more spacing between them
\usepackage{parskip}

\usepackage{mathdots}

%% Graphical
\usepackage{tikz}
\usepackage{tikz-cd}
\usetikzlibrary{automata, positioning, arrows, shapes}
\tikzset{ %https://www3.nd.edu/~kogge/courses/cse30151-fa17/Public/other/tikz_tutorial.pdf
    ->, % makes the edges directed
    >=stealth, % makes the arrow heads bold
    node distance=2cm, % specifies the minimum distance between two nodes. Change if necessary.
    every state/.style={thick, fill=gray!10}, % sets the properties for each ’state’ node
    initial text=\( \), % sets the text that appears on the start arrow
}

\usepackage{graphicx}
\usepackage{import}
\usepackage{xifthen}
\usepackage{pdfpages}
\usepackage{transparent}

%% Cancel var commands
\usepackage{cancel}

%% Wrap text around figures and other graphics
\usepackage{wrapfig}

%% lorem ipsum generator
\usepackage{lipsum}

%% Caption controls
\usepackage{caption}

%% Prog-Code typeset
\usepackage{listings}
\lstset{
    showstringspaces=false
}

%% Linked TOC
\usepackage{hyperref}

%% Get name instead of number of reference via \nameref{...}
\usepackage{nameref}

%% Table styling
\usepackage{booktabs}

%% For \FloatBarrier command
\usepackage{placeins}

%% Additional math symbols + theoretical computer science symbols
\usepackage{stmaryrd}

%% Proof trees
\usepackage{bussproofs}
\usepackage{bussproofs-extra}

%>> -------------------- Redeclare math op --------------------  <<%%
%% https://tex.stackexchange.com/questions/175251/how-to-redefine-a-command-using-declaremathoperator
\makeatletter
\newcommand\RedeclareMathOperator{%
    \@ifstar{\def\rmo@s{m}\rmo@redeclare}{\def\rmo@s{o}\rmo@redeclare}%
}
% this is taken from \renew@command
\newcommand\rmo@redeclare[2]{%
    \begingroup \escapechar\m@ne\xdef\@gtempa{{\string#1}}\endgroup
    \expandafter\@ifundefined\@gtempa
        {\@latex@error{\noexpand#1undefined}\@ehc}%
        \relax
    \expandafter\rmo@declmathop\rmo@s{#1}{#2}}
% This is just \@declmathop without \@ifdefinable
\newcommand\rmo@declmathop[3]{%
    \DeclareRobustCommand{#2}{\qopname\newmcodes@#1{#3}}%
}
\@onlypreamble\RedeclareMathOperator
\makeatother
\usepackage{titlesec}
\titleformat*{\paragraph}{\normalfont\itshape}

%%>> -------------------- Checklist --------------------  <<%%
\usepackage{pifont}
\usepackage{enumitem}
\newlist{checklist}{itemize}{2}
\setlist[checklist]{label=\(\square\)}
\newcommand{\done}{\rlap{\(\square\)}{\raisebox{1.5pt}{\large\hspace{1pt}\ding{51}}}\hspace{-2.5pt}}

%%>> -------------------- QOL Commands --------------------  <<%%

\DeclareMathOperator\N{\mathbb{N}}
\DeclareMathOperator\R{\mathbb{R}}
\DeclareMathOperator\Z{\mathbb{Z}}
\let\p\relax
\RedeclareMathOperator\P{\mathbb{P}}
\DeclareMathOperator\E{\mathbb{E}}
\renewcommand\O{\emptyset}
\DeclareMathOperator\Q{\mathbb{Q}}
\DeclareMathOperator\C{\mathbb{C}}
\DeclareMathOperator\infin{\infty}

\DeclareMathOperator\sub{\subset}
\DeclareMathOperator\sube{\subseteq}
\DeclareMathOperator\nsube{\nsubseteq}
\DeclareMathOperator\unif{\text{Unif}}

\newcommand{\howe}[1][\tau]{\preccurlyeq_{#1}^{\mathcal{H}}}
\newcommand{\howep}{\howe[\tau\times\tau']}
\newcommand{\osim}[1][\tau]{\preccurlyeq_{#1}^{\circ}}
\newcommand{\osimp}{\osim[\tau\times\tau']}
\newcommand{\asim}[1][\tau]{\preccurlyeq_{#1}}


%% https://tex.stackexchange.com/questions/85059/proof-environment-line-break-after-the-proof
\newcommand{\forceNewline}{\( \)\par\nobreak\ignorespaces}%

%% https://tex.stackexchange.com/questions/50804/explicit-space-character
\newcommand\visiblespace[1][.3em]{%
    \mbox{\kern.06em\vrule height.3ex}%
    \vbox{\hrule width#1}%
    \hbox{\vrule height.3ex}%
}
\usepackage{geometry} % Required for adjusting page dimensions and margins
\geometry{margin=2cm}

%%>> -------------------- Theorem Environments --------------------  <<%%


% \usepackage{thmtools}
% \usepackage{thm-restate}
% \usepackage[framemethod=TikZ]{mdframed}
% \mdfsetup{skipabove=1em,skipbelow=0em, innertopmargin=12pt, innerbottommargin=8pt}

% \theoremstyle{definition}

% \declaretheoremstyle[headfont=\bfseries\sffamily, bodyfont=\normalfont, mdframed={ nobreak } ]{thmgreenbox}
% \declaretheoremstyle[headfont=\bfseries\sffamily, bodyfont=\normalfont, mdframed={ nobreak } ]{thmredbox}
% \declaretheoremstyle[headfont=\bfseries\sffamily, bodyfont=\normalfont]{thmbluebox}
% \declaretheoremstyle[headfont=\bfseries\sffamily, bodyfont=\normalfont]{thmblueline}
% \declaretheoremstyle[headfont=\bfseries\sffamily, bodyfont=\normalfont, numbered=no, mdframed={ rightline=false, topline=false, bottomline=false, }, qed=\qedsymbol ]{thmproofbox}
% \declaretheoremstyle[headfont=\bfseries\sffamily, bodyfont=\normalfont, numbered=no, mdframed={ nobreak, rightline=false, topline=false, bottomline=false } ]{thmexplanationbox}


% \declaretheorem[numbered=no, style=thmgreenbox, name=Definition]{definition}
% \declaretheorem[numberwithin=chapter, style=thmredbox, name=Theorem]{theorem}
% \declaretheorem[sibling=theorem, style=thmredbox, name=Corollary]{corollary}
% \declaretheorem[sibling=theorem, style=thmredbox, name=Proposition]{prop}
% \declaretheorem[sibling=theorem, style=thmredbox, name=Lemma]{lemma}
% \declaretheorem[numberwithin=chapter, style=thmbluebox, name=Example]{example}
% \declaretheorem[sibling=example, style=thmbluebox, name=Nonexample]{nonexample}
% \declaretheorem[numbered=no, style=thmblueline, name=Remark]{remark}
% \declaretheorem[numbered=no, style=thmredbox, name=Observation]{observation}
% \declaretheorem[numbered=no, style=thmredbox, name=Postulate]{postulate}
% \declaretheorem[sibling=theorem, style=thmredbox, name=Observation]{observationnum}
% \declaretheorem[numbered=no, style=thmblueline, name=Axiom]{axiom}
% \declaretheorem[numbered=no, style=thmblueline, name=Conjecture]{conjecture}
% \declaretheorem[numbered=no, style=thmblueline, name=Notation]{notation}
% \declaretheorem[style=thmbluebox,  numbered=no, name=Exercise]{exercise}

%>> -------------------- Figures --------------------  <<%%

%% For non-tuftebook styled sheets this is almost identical to \largeinkfig
\newcommand{\inkfig}[2][1]{%
    \def\svgwidth{#1\textwidth}%
    \import{./figures}{#2.pdf_tex}%
}

\newcommand{\largeinkfig}[2][0.90]{%
    \def\svgwidth{#1\paperwidth}%
    \import{./figures}{#2.pdf_tex}%
}


\newcommand{\smallinkfig}[2][0.7]{%
    \def\svgwidth{\columnwidth}
    \resizebox{#1\textwidth}{!}{\import{./figures}{#2.pdf_tex}}
}
